
\usepackage[T1]{fontenc}
\usepackage{lmodern}
\usepackage{url}
\usepackage[svgnames]{xcolor}
\ifpdf
\usepackage{pdfcolmk}
\fi
%% check if using xelatex rather than pdflatex
\ifxetex
\usepackage{fontspec}
\fi
\usepackage{graphicx}
%%\usepackage{hyperref}
%% drawing package
\usepackage{tikz}
%% for dingbats
\usepackage{pifont}
\providecommand{\HUGE}{\Huge}% if not using memoir
\newlength{\drop}% for my convenience
%% select a (FontSite) font by its font family ID
\newcommand*{\FSfont}[1]{\fontencoding{T1}\fontfamily{#1}\selectfont}
%% if you don’t have the FontSite fonts either \renewcommand*{\FSfont}[1]{}
%% or use your own choice of family.
%% select a (TeX Font) font by its font family ID
\newcommand*{\TXfont}[1]{\fontencoding{T1}\fontfamily{#1}\selectfont}
%% Generic publisher’s logo
\newcommand*{\plogo}{\fbox{$\mathcal{PL}$}}
%% Some shades
%%%% Additional font series macros
\makeatletter
%%%% light series
%% e.g., kernel doc, section s: line 12 or thereabouts
\DeclareRobustCommand\ltseries
{\not@math@alphabet\ltseries\relax
\fontseries\ltdefault\selectfont}
%% e.g., kernel doc, section t: line 32 or thereabouts
\newcommand{\ltdefault}{l}
%% e.g., kernel doc, section v: line 19 or thereabouts
\DeclareTextFontCommand{\textlt}{\ltseries}
% heavy(bold) series
\DeclareRobustCommand\hbseries
{\not@math@alphabet\hbseries\relax
\fontseries\hbdefault\selectfont}
\newcommand{\hbdefault}{hb}
\DeclareTextFontCommand{\texthb}{\hbseries}
\makeatother


\newcommand*{\titleGM}{\begingroup% Gentle Madness
\drop = 0.1\textheight
\vspace*{\baselineskip}
\vfill
\hbox{%
\hspace*{0.2\textwidth}%
\rule{1pt}{\textheight}
\hspace*{0.05\textwidth}%
\parbox[b]{0.75\textwidth}{
\vbox{%
\vspace{\drop}
{\noindent\HUGE\bfseries Some\\[0.5\baselineskip]
Conundrums}\\[2\baselineskip]
{\Large\itshape Puzzles for the Mind}\\[4\baselineskip]
{\Large THE AUTHOR}\par
\vspace{0.5\textheight}
{\noindent The Publisher}\\[\baselineskip]
}% end of vbox
}% end of parbox
}% end of hbox
\vfill
\null
\endgroup}