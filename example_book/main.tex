\documentclass{book}
% Copyright 2022 Riley Hanus, PhD
% Unauthorized copying and use of this file, via any medium is strictly prohibited
% Proprietary and confidential
% Written by Riley Hanus <hanusriley@gmail.com>, November 2022

\usepackage[framemethod=TikZ]{mdframed} 
\usepackage{verbatim}
\usepackage{hyperref}
\usepackage{xstring}
\usepackage{listofitems}
\usepackage{substr}
\usepackage{cprotect}
% \newcounter{iframe}[chapter]\setcounter{iframe}{0}
% \renewcommand{\theiframe}{\arabic{chapter}.\arabic{iframe}}

\newenvironment{iframe}[1][]{%
% \refstepcounter{iframe}%
\ifstrempty{#1}%
{\mdfsetup{%
frametitle={%
\tikz[baseline=(current bounding box.east),outer sep=0pt]
\node[anchor=east,rectangle,fill=blue!20]
{\strut Digital Content};}}
}%
{\mdfsetup{%
frametitle={%
\tikz[baseline=(current bounding box.east),outer sep=0pt]
\node[anchor=east,rectangle,fill=blue!20]
{\strut Digital Content~:~#1};}}%
}%
\mdfsetup{innertopmargin=10pt,linecolor=blue!20,%
linewidth=2pt,topline=true,%
frametitleaboveskip=\dimexpr-\ht\strutbox\relax
}
\begin{mdframed}[]}{\end{mdframed}}

\newcommand{\parseiframe}[1]{\setsepchar{"}\readlist\iframelist{#1}
}

\newcommand{\iframeurl}[2]{
    \parseiframe{#1}
    \foreachitem \i \in \iframelist{
        \IfSubStringInString{://www.}{\i}{ \let\result=\i}{}
    }
    \href{\result}{#2}
}

\newcommand{\InsertIframe}[4]{
    % #1: html code for <iframe>
    % #2: display name for hyperlink in 'Digital Content box' in pdf
    % #3: caption
    % #4: type of media for 'Digital Content: xxx'
   \begin{iframe}[#4]
      \iframeurl{#1}{#2} | #3
   \end{iframe}
}

%%%%%%%%%%%%%%%%%%%%%%%%%%%%%%%%%%%%%%%%%%%%%%%%%%%%%%%%%%%%%%%%%%%%%%%%%%%%%%%%
% calchub environment: for embedding a calchub workspace                       %
% texedbook compatible: html=embedded work space, pdf=multimedia box w/ href   %
%%%%%%%%%%%%%%%%%%%%%%%%%%%%%%%%%%%%%%%%%%%%%%%%%%%%%%%%%%%%%%%%%%%%%%%%%%%%%%%%
\newenvironment{calchub}
    {\begin{iframe}[CalcHub Workspace]
    }
    {\end{iframe}
    }

\newcommand{\InsertCalchub}[3]{ 
   % 1st input: href 
   % 2nd input: display name
   % 3rd input: description
   \begin{calchub}
      \href{#1}{#2} | #3
   \end{calchub}
} 


%%%%%%%%%%%%%%%%%%%%%%%%%%%%%%%%%%%%%%%%%%%%%%%%%%%%%%%%%%%%%%%%%%%%%%%%%%%%%%%%
% video environment: for embedding videos                                      %
% texedbook compatible: html=embedded video, pdf=multimedia box w/ href        %
%%%%%%%%%%%%%%%%%%%%%%%%%%%%%%%%%%%%%%%%%%%%%%%%%%%%%%%%%%%%%%%%%%%%%%%%%%%%%%%%
\newenvironment{youtube}
   {\begin{iframe}[Video]
   }
   {\end{iframe}
   }

\newcommand{\InsertYouTube}[3]{ 
   % 1st input: href   
   % 2nd input: display name 
   % 3rd input: description
   \begin{youtube}
      \href{#1}{#2} | #3
   \end{youtube}
} 

%%%%%%%%%%%%%%%%%%%%%%%%%%%%%%%%%%%%%%%%%%%%%%%%%%%%%%%%%%%%%%%%%%%%%%%%%%%%%%%%
% python coding environment: for coding practice                               %
% texedbook compatible: html=embedded trinket, pdf=multimedia box w/ href      %
%%%%%%%%%%%%%%%%%%%%%%%%%%%%%%%%%%%%%%%%%%%%%%%%%%%%%%%%%%%%%%%%%%%%%%%%%%%%%%%%


\newenvironment{trinket}
   {\begin{iframe}[Python coding environment]
   }
   {\end{iframe}
   }

\newcommand{\InsertTrinket}[3]{ 
   % 1st input: href   
   % 2nd input: display name 
   % 3rd input: description
   \begin{trinket}
      \href{#1}{#2} | {#3}
   \end{trinket}
} 


%%%%%%%%%%%%%%%%%%%%%%%%%%%%%%%%%%%%%%%%%%%%%%%%%%%%%%%%%%%%%%%%%%%%%%%%%%%%%%%%
% panopto environment: for viewing videos hosted by panopto                    %
% texedbook compatible: html=embedded iframe, pdf=multimedia box w/ href       %
%%%%%%%%%%%%%%%%%%%%%%%%%%%%%%%%%%%%%%%%%%%%%%%%%%%%%%%%%%%%%%%%%%%%%%%%%%%%%%%%


\newenvironment{panopto}
   {\begin{iframe}[Video]
   }
   {\end{iframe}
   }

\newcommand{\InsertPanoptoVideo}[3]{ 
   % 1st input: href   
   % 2nd input: display name 
   % 3rd input: description
   \begin{panopto}
      \href{#1}{#2} | {#3}
   \end{panopto}
} 

% Mathjax equation referencing
\newcommand\mjref[1]{\ref{#1}}



\title{Example Book for Texedbook}
\author{Riley Hanus}

\begin{document}
 
\maketitle

Texedbook is a tool for publishing articles and textbooks online without the need to learn html, css, and javascript. The author writes and compiles the content in latex and runs texedbook, which generates a collection of html and css files that can be directly published online. In addition to supporting most of the native latex features, texedbook provides tools for embbedding digital and interactive content such as videos, code editors and compilers, etc. 

This article will demonstrate  document features that are explicitly supported by texedbook.  First the native latex capabilities will be demonstrated including Citations, Equations, Figures, Lists, and Tables. Then the texedbook specific features will be presented which allow the author to embbed digial content (anything that can be contained in an iframe) straight from the latex document. 

\chapter{Motivation for texedbook project}

When writing an article or textbook, especially one that is technical in nature, the need for proper handling of standard document features is glaring. Without a framework to efficiently write and cross-reference equations, figures, tables, etc. in real time, writing anything with technical substanance becomes impossible. Latex, despite its quarks, is a very good framework to manage these critical writing tools.

\chapter{Citations}
The native bibliography features are maintained \cite{Ohno2007}. 

\chapter{Equations}
Equations are an inherently tricky problem for digital publishing. The core of the problem lies in the fact that html was designed around the standard alpha numeric alphabet, and math requires a wider range of complex symbols and typesetting. The default useage of texedbook leverages mathjax, allowing all of the native latex equations, that the author spent so much time perfecting, is reliably reproduced in the html output.


\chapter{Inline equations}
In equations can be included using both methods: $n\lambda=2d \sin \theta$ or \( n\lambda=2d \sin \theta \). In addition unicode characters can be directly writen in the latex, and they rendered in the latex and preserved through into the html output, 
Γσµ and such symbols §¶∂∇ a.

\bibliographystyle{plain}
\bibliography{sample}

\end{document}

